\frontmatter
%███████████████████████████████████████████████████████████████████
%███████████████████████████████████████████████████████████████████
%███████████████████████████████████████████████████████████████████
\chapter{Intro}

Newtonian physics serves as an excellent approximation at low speeds, accurately predicting the behavior of objects in our everyday world, but it gives way to different effects as we approach the speed of light. It turns out that an \hyperlink{def-observer}{observer} moving at high speed relative to another \hyperlink{def-observer}{observer} would notice lengths contract, time run slower, and \hyperlink{def-event}{event}s that were \hyperlink{def-simultaneity}{simultaneous} to the other \hyperlink{def-observer}{observer}, not being \hyperlink{def-simultaneity}{simultaneous} to them. There is however one thing that remains constant, the speed that light travels relative to each \hyperlink{def-observer}{observer} is the same. To understand how this can be, we have to redefine our understanding of time and space, with special relativity.

Special relativity is vital for particle, nuclear, and astrophysics. It is needed for calculating the power output when it comes to nuclear fusion and fission reactions or measuring the speed of galaxies and stars from the relativistic effects on the frequency of their emitted light. It is also needed for precision timekeeping for the likes of the fast-moving GPS satellites which need it for accurately calculating positions.

Despite it being well-established and an essential part of fundamental physics, special relativity continues to confuse even those well-versed in physics and mathematics and it can be just as confusing to visualize and understand as quantum mechanics. The theory requires letting go of some deeply held intuitions about the absolute nature of space and time, that differ from our everyday experience of the physical world.

This handbook offers an intuitive, visual, and comprehensive overview of the fundamentals of special relativity. It aims to maintain simplicity and avoids unnecessary abstraction in both visual and conceptual explanations. Time in diagrams will be shown either through the use of animations or multiple diagrams set on a timeline, as this is a more natural way of viewing the physics than having time drawn on one of the axes, like in space-time diagrams. Though external resources will be provided at the end for the more abstract space-time visualizations, as well as other resources where you can test out the concepts and mathematics you will learn in this handbook.

We will begin with a non-mathematical visual overview of the concepts in special relativity, building step-by-step an intuitive understanding of the core ideas, and providing a solid foundation to develop the essential mathematical framework. The math will start by deriving how space and time coordinates work in special relativity, followed by velocities, which lead to time dilation, length contraction, relativity of simultaneity, Doppler effect, and energy-momentum relationship. In addition to these, we will also be exploring some interesting topics not typically covered in the standard resources for special relativity such as a source's light being relativistically aberrated and beamed, with a delay in light signals leading to the retarded source positions and retarded fields.
\vspace{1cm} \newline
Now let us begin!

%the Liénard-Wiechert-Potentials, maybe quaternions in SR if you are lucky, and so on.
% Your moma didn't raise you right, so now its my turn to whip you into shape. By the end of this book you will also be able to understand yourself a little more, (because there will be a subsection on retarded jerks). You already have something in common with this topic of relativity, cause you are special!... The only thing that can undo Lorentz length contraction is your moma
% This book is not an inertial frame of reference unless in free fall (but gravity and that is for a different day). Damage caused to this book by allowing it to enter a short lived inertial frame of reference before making an abrupt exit from it, is not covered by our returns policy (do not throw book at ground!). \\

\newpage