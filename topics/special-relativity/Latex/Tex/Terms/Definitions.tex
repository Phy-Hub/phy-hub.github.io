
%*** observered vs apparent vs perceived vs retarded when it comes to velocity, time, arrival time, position


\newgeometry{top=1in, bottom=1in, left=1in, right=1in}
\setlength{\headwidth}{\textwidth}
%███████████████████████████████████████████████████████████████████
%███████████████████████████████████████████████████████████████████
\section{Definitions}
\begin{multicols}{2}

%██████████████████
\subsection*{\underline{Mathematical Framework}}

%
\term{def-Reference-frame}{Reference frame}{reference frame,frame of reference}{
An abstract coordinate system, consisting of three spatial and one time axis. The origin, orientation, and scale of its axis are specified by a set of points in space. The purpose of it is to provide a standardized means of defining the position and orientation of objects at any instant in time.}

%
\term{def-Inertial-reference-frame}{Inertial reference frame}{Inertial reference frame, Inertial frame of reference}{
A reference frame that is not undergoing any acceleration. An observer's reference frame is inertial if they do not feel a force acting on them.}

%
\term{def-proper-frame}{Proper/Rest frame}{proper frame,rest frame}{
This is the reference frame of the observer or object itself.}

%
\term{def-Primed-Frame}{Primed frame}{Primed frame,Prime frame}{
A reference frame that is moving relative to the current frame.}

%
\term{def-initial-Frame}{Initial frame}{initial frame}{
The first reference frame that we are comparing the current or primed reference frame to.}

%
\term{def-Moving-Frame}{Moving frame}{moving frame}{
An inertial reference frame that is moving relative to the rest frame.}

%
\term{def-frame-velocity}{frame velocity}{frame velocity}{
The velocity of a reference frame relative to the current reference frame.}

%
\term{def-transform}{Frame transform}{frame transform,frame transformation,transform,transformation}{
The changing of coordinates and other quantities in a reference frame into their corresponding values in another reference frame.}

%
\term{def-galilean-transform}{Galilean/Classical frame transform}{galilean,galilean transform,classical transform,Galilean frame transform,classical frame transform,classical tranformation}{
Transformation of coordinates and physical quantities according to Newtonian physics between two frames of reference.}

%
\term{def-lorentz-transform}{Relativistic frame transform}{Relativistic transform,Relativistic frame transform,Relativistic tranformation,lorentz transform,lorentz transformation}{
The changing of the spatial coordinates, the time coordinate, and other physical quantities into their corresponding values in a different inertial reference frame using equations from special relativity.}

%
\term{def-observer}{Observer}{observer}{
An entity/thing with an associated frame of reference/system of coordinates, with respect to which you can measure an object's position, orientation, shape, signal emission and other properties.}

%
\term{def-event}{Event}{event}{
An occurrence that can be specified by a unique combination of the three spatial and one time coordinates. These four coordinates, together, form a point in spacetime.}

%
\term{def-simultaneity}{Simultaneity}{simultaneous,simultaneity,simultaneously}{
Within a reference frame, two or more events are simultaneous if they occur at the same time within it. Classically, if any two events are simultaneous in one reference frame, they are also simultaneous in another. This is not true in special relativity, and the order of events can depend on the frame of reference.}

%
\term{def-retarded-position}{Retarded position}{retarded position}{
The previous position of a source, that an observer currently sees it at, due to the time delay between when a moving source emitting the signal and when the observer detects it.}

%██████████████████
\subsection*{\underline{The Mechanics}}

%
\term{def-aether}{Aether}{aether}{
A proposed medium that filled the vacuum of space that light propagated through.}

%
\term{def-vacuum}{Vacuum}{vacuum}{
Space that contains no matter.}

%
\term{def-length-contraction}{Lorentz length contraction}{lorentz length contraction,length contraction}{
The shortening of a moving object along the direction of its motion relative to an observer (compared to the length of object at rest).}

%
\term{def-time-dilation}{Time dilation}{time dilation}{
The slowing of the passing of time for objects in motion relative to an observer, compared to the passing of time for the observer.}

%
\term{def-doppler-effect}{Doppler effect}{doppler effect}{
The change of a wave's frequency due to the motion of its source relative to an observer. This is due to the bunching up of the wave in the direction of movement of the source and the stretching of the wave in the opposite direction, as well as the dilation of time between the emission of the crests of the emitted wave.}

%
\term{def-aberration}{Aberration of light}{aberration,aberrated,aberrate,aberrating,aberrates}{
The changing of the direction of the propagation of light when changing to another reference frame.}

%
\term{def-relativistic-beaming}{Relativistic beaming}{relativistic beaming,beaming}{
The accumulation of the Doppler effect and aberration of light, and how it affects the amount of light and its frequency emitted at different angles from a moving source. This effect can be extended to any wave or field emitted at the speed of light from a moving source.}

%
\term{def-retarded-field}{Retarded field}{retarded field}{
A moving source's field, where each point in the field has had its field propagated at the speed of light from the source's retarded position (i.e. the previous position of the source where it had emitted that part of the field from).}

%
\term{def-field}{Field}{field}{
A physical quantity that has a value (or a set of values) assigned to every point in space and time.}

%
\term{def-light-year}{Light year}{light year}{
The distance light travels in one year (in a vacuum).}

%██████████████████
\subsection*{\underline{Definitions for Later Chapters}}

%
\term{def-lorentz-invariant}{Lorentz Invariant/Frame independent}{invariant,lorentz invariant,transform invariant,transformation invariant,frame independent}{
A quantity or law of physics is Lorentz Invariant/Frame independent, if it remains unchanged when transforming coordinates and other quantities to any other inertial reference frame.}

%
\term{def-postulates}{Postulates}{postulate}{
A statement that is assumed to be true and is used as a basis for reasoning for a theory.}

%
\term{def-flux}{Flux}{flux}{
A measure of how much of a quantity is passing through a given surface in a given time (flowrate through a surface)}

%
\term{def-apparent-velocity}{Perceived/Apparent Velocity}{perceived velocity,apparent velocity}{
The velocity an object appears to have due to the time delay in the signal of light from it, as the time delay from each position along a sources path is different due to having different distances the signal needs to travel to observer (It is the consequence of the time difference in the propagation time of the light signal from the different positions of a source's path).}

%
\term{def-rest-mass}{Rest mass}{Rest mass,proper mass}{
The mass of a particle in its own rest frame}

%
\term{def-rest-mass-energy}{Rest mass energy}{Rest mass energy,rest energy, proper energy}{
The energy a particle has due to its rest mass (given by its rest mass times speed of light squared)}

\end{multicols}
\restoregeometry
\setlength{\headwidth}{\textwidth}