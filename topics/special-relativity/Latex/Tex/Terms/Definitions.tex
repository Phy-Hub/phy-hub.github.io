%███
% need precisely one new line
%███

\section{Definitions}

\subsection{Mathematical Framework}%███

\linesep

% {reference frame} {frame of reference}
\noindent \hypertarget{def-Reference-frame}{\textbf{Reference frame:}}
An abstract coordinate system that is taken to be at rest consisting of three spatial and one time axis. The origin, orientation, and scale of its axis are specified by a set of points in space. The purpose of it is to provide a standardized means of defining the position and orientation of objects at any instant in time.

\linesep

% {Inertial reference frame} {Inertial frame of reference}
\noindent \hypertarget{def-Inertial-reference-frame}{\textbf{Inertial reference frame:}}
A reference frame that is not undergoing any acceleration. An observer's reference frame is inertial if they do not feel a force acting on them.

\linesep

% {proper frame} {rest frame}
\noindent \hypertarget{def-proper-frame}{\textbf{Proper/Rest frame:}}
This is the reference frame of the observer or object itself.

\linesep

%%% initial frame

% {Primed frame} {Prime frame}
\noindent \hypertarget{def-Primed-Frame}{\textbf{Primed frame:}}
A reference frame that is moving relative to the current frame.

\linesep

% {frame velocity}
\noindent \hypertarget{def-frame-velocity}{\textbf{frame velocity:}}
The velocity of a reference frame relative to the current reference frame.

\linesep

% {frame transform} {frame transformation} {transform} {transformation}
\noindent \hypertarget{def-transform}{\textbf{Frame transform:}}
The changing of coordinates and other quantities in a reference frame into their corresponding values in another reference frame.

\linesep

% {galilean} {galilean transform} {classical transform} {Galilean frame transform} {classical frame transform} {classical tranformation}
\noindent \hypertarget{def-galilean-transform}{\textbf{Galilean/Classical frame transform:}}
Transformation of coordinates and physical quantities according to Newtonian physics between two frames of reference.

\linesep

% {Relativistic transform} {Relativistic frame transform} {Relativistic tranformation} {lorentz transform} {lorentz transformation}
\noindent \hypertarget{def-lorentz-transform}{\textbf{Relativistic frame transform:}}
The changing of the spatial coordinates, the time coordinate, and other physical quantities into their corresponding values in a different inertial reference frame using equations from special relativity.

\linesep

% {observer}
\noindent \hypertarget{def-observer}{\textbf{Observer:}}
An entity/thing with an associated frame of reference/system of coordinates, with respect to which you can measure objects' position, orientation, and other properties.

\linesep

% {event}
\noindent \hypertarget{def-event}{\textbf{Event:}}
An occurrence that can be specified by a unique combination of the three spatial and one time coordinates. These four coordinates, together, form a point in spacetime.

\linesep

% {simultaneous} {simultaneity} {simultaneously}
\noindent \hypertarget{def-simultaneity}{\textbf{Simultaneity:}}
Within a reference frame, two or more events are simultaneous if they occur at the same time within it. Classically, if any two events are simultaneous in one reference frame, they are also simultaneous in another. This is not true in special relativity, and the order of events can depend on the frame of reference.

\linesep

% {retarded position}
\noindent \hypertarget{def-retarded-position}{\textbf{Retarded position:}}
The previous position of a field's source, that an observer sees it at, due to the time delay between when a moving source emits a field and when the observer detects that field.

\linesep

\subsection{The Mechanics}%██████████████████

\linesep

% {aether}
\noindent \hypertarget{def-aether}{\textbf{Aether:}}
A proposed medium that filled the vacuum of space that light propagated through.

\linesep

% {vacuum}
\noindent \hypertarget{def-vacuum}{\textbf{Vacuum:}}
Empty space with no matter and negligible amount of effects from fields.

\linesep

% {lorentz length contraction} {length contraction}
\noindent \hypertarget{def-length-contraction}{\textbf{Lorentz length contraction:}}
The shortening of an object along the direction of its motion relative to an observer (relative to the object at rest).

\linesep

% {time dilation}
\noindent \hypertarget{def-time-dilation}{\textbf{Time dilation:}}
The slowing of the passing of time for objects in motion relative to an observer, compared to the passing of time for the observer.

\linesep

% {doppler effect}
\noindent \hypertarget{def-doppler-effect}{\textbf{Doppler effect:}}
The change of a wave's frequency due to the motion of its source relative to an observer. This is due to the bunching up of the wave in the direction of movement of the source and the stretching of the wave in the opposite direction, as well as the dilation of time between two parts of the emitted wave.

\linesep

% {aberration} {aberrated} {aberrate} {aberrating} {aberrates}
\noindent \hypertarget{def-aberration}{\textbf{Aberration of light:}}
The changing of the direction of the propagation of light when moving into another reference frame.

\linesep

% {relativistic beaming} {beaming}
\noindent \hypertarget{def-relativistic-beaming}{\textbf{Relativistic beaming:}}
The accumulation of the Doppler effect and aberration of light, and how it affects the amount of light and its frequency emitted at different angles from a moving source. This effect can be extended to any wave or field emitted at the speed of light from a moving source.

\linesep

% {retarded field}
\noindent \hypertarget{def-retarded-field}{\textbf{Retarded field:}}
A moving source's field, where at each point has been propagated at the speed of light from the retarded position of the source...

\linesep

% {field}
\noindent \hypertarget{def-field}{\textbf{Field:}}
...

\linesep

% {light year}
\noindent \hypertarget{def-light-year}{\textbf{Light year:}}
...

\linesep

\subsection{Definitions for Later}%██████████████████

% {invariant} {lorentz invariant} {transform invariant}
\noindent \hypertarget{def-lorentz-invariant}{\textbf{Transform Invariant:}}
...

\linesep

% {postulate} {postulates}
\noindent \hypertarget{def-postulates}{\textbf{Postulates:}}
...

\linesep

% {flux}
\noindent \hypertarget{def-flux}{\textbf{Flux:}}
describes the rate at which a quantity flows across a surface (flowrate per unit area)

\linesep

% {perceived velocity} {retarded velocity}
\noindent \hypertarget{def-retarded-velocity}{\textbf{Perceived/Retarded Velocity:}}
...